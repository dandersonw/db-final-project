\documentclass{article}

\usepackage{amsthm}
\usepackage{amsmath}
\usepackage{cite}
\usepackage{listings}
\usepackage{multicol}
\usepackage{url}

\setlength{\parindent}{4em}
\setlength{\parskip}{1em}


\begin{document}

\pagenumbering{gobble}

\begin{center}
  \textbf{Project Progress Report}

  \textit{Derick Anderson and Harshitha Bhaskar, Blackboard: AndersonBhaskar}
\end{center}

\section*{Idea}

The core idea of the project is to develop a journaling tool for managing your
reading. Functionality will include, for example, the ability to track your
progress through books from want-to-read to finished, write reviews of books,
and organize books by author and series. To support said functionality will
require a data model for books, authors, reading lists, etc.

We love to read, but lose track of all the information to do with our
reading. It is great to be able to look forward to the things you want to read,
look back on the things you’ve read, and organize your thoughts about your
reading.

\section*{Response to Feedback/Questions}

We have decided to use MySQL instead of SQLite.

Derick has extensive experience with Python,
but no experience with MySQL
except that gained through this class.
% TODO Harshitha: what is your experience?

A small set of data will be manually curated to demonstrate the functionality of
the system.

\section*{UML and EER Diagrams}

\section*{Sample User Interaction}

% TODO Harshitha: fill in here

\section*{Technical Specifications}



\end{document}
%%% Local Variables:
%%% mode: latex
%%% TeX-master: t
%%% End:
